\documentclass{standalone}

\usepackage{tikz}
\usetikzlibrary{angles,quotes}
\usepackage{amsmath,amssymb,amsfonts}

\usepackage{pgfplots}
\definecolor{darkgreen}{rgb}{0.0, 0.42, 0.24}
\definecolor{amethyst}{rgb}{0.6, 0.4, 0.8}

\usetikzlibrary{calc}
\pgfplotsset{width=10cm,compat=1.9}
\usepgfplotslibrary{fillbetween}

\newcommand{\anglearc}[6]{%
    \begin{scope}
      \coordinate (AngleArcA) at (axis cs:#1);
      \coordinate (AngleArcB) at (axis cs:#2);
      \coordinate (AngleArcC) at (axis cs:#3);
      \path [clip] (AngleArcA) -- (AngleArcB) -- (AngleArcC) -- cycle;
      \node (AngleArcCircle) [#5,circle,minimum size=#4] at (axis cs:#2) {};
    \end{scope}
      \coordinate (AngleArcA1) at (intersection of AngleArcB--AngleArcA and AngleArcCircle);
      \coordinate (AngleArcC1) at (intersection of AngleArcB--AngleArcC and AngleArcCircle);
    
%
}

\begin{document}
\begin{tikzpicture}
       \begin{axis}[axis lines=center,
       xmin=-11,
       xmax=11,
       ymin=-11,
       ymax=11,
       axis equal,
       xlabel=$x$,
       ylabel=$y$,
       yticklabels=false,
       xticklabels=false]
           
           \draw[color=gray,dashed, rotate around={40:(axis cs:0,0)}] (axis cs: 0,0) ellipse (7.8 and 5);
           \fill (axis cs: 6,5) circle (0.25);
            \fill (axis cs: -6,-5) circle (0.25);
           \node[scale=0.6] at (axis cs:7,5){$m_1$};
           \node[scale=0.6] at (axis cs:-7,-5){$m_2$};
           \draw[->,color=black](axis cs:0,0)--(axis cs:5.85,4.9);
            \draw[->,color=black](axis cs:0,0)--(axis cs:-5.85,-4.9);
            
            \anglearc{6,5}{0,0}{5,0}{1.5cm}{draw}{$\omega t$}
            \node[scale=0.6] at (axis cs:1.35,0.5){$\omega t$};
            \node[scale=0.6] at (axis cs:2.9,3.5){$r_1$};
             \node[scale=0.6] at (axis cs:-2.9,-3.5){$r_2$};
            
            \addplot[color=gray,dashed,samples=3] coordinates {(6,0)(6,5)};
            \node[scale=0.6] at (axis cs:7.5,2.5){$r_1\cos{\omega t}$};
            \addplot[color=gray,dashed,samples=3] coordinates {(6,5)(0,5)};
            \node[scale=0.6] at (axis cs:3,5.5){$r_1\sin{\omega t}$};
       \end{axis}
    \end{tikzpicture}
\end{document}